\section{Map Method} 

\begin{itemize}
    \item The complexity of digital logic gates depends on the algebraic expression of a Boolean function. 
    \item While truth tables provide a unique representation, Boolean expressions can have multiple equivalent forms. 
    \item Algebraic simplification is possible but lacks systematic rules. 
    \item The \color{blue} Karnaugh map (K-map) \color{black} method offers a visual way to minimize Boolean functions by grouping minterms efficiently. It allows users to derive simpler expressions, reducing the number of gates and inputs required in a circuit. 
    \item The minimized expressions follow either the sum of products or product of sums form. Multiple minimal expressions may exist, and any valid solution is acceptable.
\end{itemize}

\subsection{Two Variable k-map}
\begin{itemize}
    \item A two-variable Karnaugh map (K-map) consists of four squares, each representing a minterm of the two variables  $x$  and  $y$ . 
    \item The map visually organizes these \textbf{minterms}, with rows and columns corresponding to the variable values. By marking the squares containing minterms of a given Boolean function, the K-map provides an intuitive way to represent any of the 16 possible Boolean functions for two variables. 
    \item For example, the function  $xy$  is represented by marking the minterm  $m_3$ , while  $x + y $ is represented by marking minterms  $m_1, m_2$,  and  $m_3$ . This visualization helps in understanding Boolean expressions and their simplifications.
\end{itemize}

\subsection{Three Variable k-Map}

\begin{itemize}
    \item A three-variable Karnaugh map (K-map) consists of eight squares, each representing a minterm of three binary variables. 
    \item The minterms are arranged in a sequence similar to the \color{blue} Gray code \color{black} , ensuring that only one variable changes between adjacent squares. Each square corresponds to a unique minterm, with variables labeled to indicate their presence or complement.\\\\
\end{itemize}

\begin{itemize}
    \item The key property of the K-map is that adjacent squares differ by only one variable, allowing for Boolean simplifications. 
    \item When two adjacent minterms are summed, the differing variable is eliminated, reducing the expression. 
    \item For example, summing minterms  $m_5$  and  $m_7$  simplifies to  $xz$ . This property makes the K-map a powerful tool for minimizing Boolean functions efficiently.
\end{itemize}
