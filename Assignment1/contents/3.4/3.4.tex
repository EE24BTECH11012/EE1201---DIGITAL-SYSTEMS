\section{\textbf{Algebraic Simplification of Switching Expressions}}

\begin{enumerate}[label=\textbf{\textdagger}]
\item Simplification of switching Boolean expressions algebraically are done based on the following properties - \\
\begin{align}
	XY + XY^{\prime} &= X  \text{ Combining terms } \label{eq:1}\\
	X + XY &= X \text{ Eliminating terms } \label{eq:2}\\
	X + X^{\prime} Y &= X + Y \text{ Eliminating literals } \label{eq:3}
\end{align}
\item Simplification can, sometimes, also be done by adding some redundant terms like, 
\begin{align}
	\text{ Addition of } xx^{\prime} \label{eq:4}\\
	\text{ Multiplying by } x + x^{\prime} \label{eq:5}\\
	\text{ Adding } yz \text{ to } xy + x^{\prime} z \label{eq:6}
\end{align}
\item Examples :
	\begin{enumerate}
		\item WX + XY + X^{\prime} Z^{\prime} + WY^{\prime} Z^{\prime} \\
			\begin{align}
				F &= WX + XY + X^{\prime} Z^{\prime} + WY^{\prime} Z^{\prime} \\
				&= WX + XY + X^{\prime} Z^{\prime} + WY^{\prime} Z^{\prime} + WZ^{\prime} \text{ by CONSENSUS theorem } \\
				&= WX + XY + X^{\prime} Z^{\prime} WZ^{\prime} \text{ from  } \eqref{eq:2} \\
				&= WX + XY + X^{\prime} Z^{\prime} \text{ from } \eqref{eq:2}
			\end{align}
	\end{enumerate}
\end{enumerate}


